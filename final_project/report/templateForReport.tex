% ****** Start of file templateForReport.tex ******

% TeX'ing this file requires that you have all prerequisites
% for REVTeX 4.1 installed
%
% See the REVTeX 4 README file
% It also requires running BibTeX. The commands are as follows:
%
%  1)  latex templateForReport.tex
%  2)  bibtex templateForReport
%  3)  latex templateForReport.tex
%  4)  latex templateForReport.tex
%
\documentclass[%
 reprint,
%superscriptaddress,
%groupedaddress,
%unsortedaddress,
%runinaddress,
%frontmatterverbose,
%preprint,
%showpacs,preprintnumbers,
%nofootinbib,
%nobibnotes,
%bibnotes,
 amsmath,amssymb,
 aps,
%pra,
%prb,
%rmp,
%prstab,
%prstper,
%floatfix,
]{revtex4-1}

\usepackage{graphicx}% Include figure files
\usepackage{dcolumn}% Align table columns on decimal point
\usepackage{bm}% bold math
\usepackage{hyperref}% add hypertext capabilities
\usepackage{natbib}
\bibliographystyle{abbrvnat}

%\usepackage[mathlines]{lineno}% Enable numbering of text and display math
%\linenumbers\relax % Commence numbering lines

%\usepackage[showframe,%Uncomment any one of the following lines to test
%%scale=0.7, marginratio={1:1, 2:3}, ignoreall,% default settings
%%text={7in,10in},centering,
%%margin=1.5in,
%%total={6.5in,8.75in}, top=1.2in, left=0.9in, includefoot,
%%height=10in,a5paper,hmargin={3cm,0.8in},
%]{geometry}

\begin{document}

\title{Exploring \textsc{Monte-Carlo}-integration techniques in Bayesian model selection}% Force line breaks with \\
%\thanks{A footnote to the article title}%

\author{Jakob Krause}
 \homepage{http://www.github.com/krausejm}
 \email{krause@hiskp.uni-bonn.de}
\author{Dominic Schüchter}
 \homepage{http://www.github.com/dschuechter}
 \email{dschuechter@uni-bonn.de}

\date{\today}% It is always \today, today,
             %  but any date may be explicitly specified

\begin{abstract}
  An article usually includes an abstract, a concise summary of the work
  covered at length in the main body of the article.
  \begin{description}
  \item[Usage]
    Secondary publications and information retrieval purposes.
  \item[Structure]
    You may use the \texttt{description} environment to structure your abstract;
    use the optional argument of the \verb+\item+ command to give the category of each item.
  \end{description}
\end{abstract}
\maketitle

%\tableofcontents

\section{\label{sec:level1}Introduction}
We mainly write everything that is in \cite{sivia}. Here an awesome introduction will form
\section{Theory}

\subsection{Bayes' Theorem}
Here we have to write cool stuff and so on about bayes theorem.
\section{Methods}
Here we write up the used algorithms.
\subsection{Monte-Carlo integration}
Here we will explain Monte-Carlo sampling, that is \emph{Sequential Monte Carlo} and therein \textsc{Metropolis-Hastings}. its probably better to put these two subsections in separate sections.

\section{Examples}

\subsection{Betabinomial example (coin flip)}
Let us now consider as a starting example, the flipping of a two-sided coin, i.e. an experiment where we can measure either heads (H) or tails (T) with $50\%$ probability, respectively. This, while simple, allows us an intuitive approach to Bayesian inference and model selection as well as to the MCMC techniques discussed before. Furthermore is this example easily altered to many real-life problems, such as birth rates, $\dots$, or anything with the option of either success or failure.
\subsubsection{Analytical approach?}
Assume we throw a coin 20 times. We observe 6 H and 14 T. "Is this a fair coin?" might be a question to ask yourself since the bias in outcome is quite large. Naively expecting a fair coin we could assign a \emph{prior} to the probability of heads $\theta$ as centred around $0.5$, so for example a gaussian with $\mu=0.5,\sigma=0.1$.  
\subsubsection{Numerical approach}


\subsection{Fitting a polynomial of unknown degree}
\subsubsection{Analytical approach?}
\subsubsection{Numerical approach}

\section{Discussion}

\section{Summary}


\bibliography{refs}
\end{document}
%
% ****** End of file templateForReport.tex ******
